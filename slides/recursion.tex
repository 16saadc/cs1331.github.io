\documentclass{beamer}

\newcommand{\lesson}{Recursion}

\author[Chris Simpkins]
{Christopher Simpkins \\\texttt{chris.simpkins@gatech.edu}}
\institute[Georgia Tech] % (optional, but mostly needed)

\date[CS 1331]{}


\newcommand{\course}{Introduction to Object-Oriented Programming}
\subject{\course}
\title[\lesson]{\course}
\subtitle{\lesson}

\author[CS 1331]
{Christopher Simpkins \\\texttt{chris.simpkins@gatech.edu}}
\institute[Georgia Tech]

\date[]{}

\newcommand{\link}[2]{\href{#1}{\textcolor{blue}{\underline{#2}}}}
\newcommand{\code}{http://www.cc.gatech.edu/~simpkins/teaching/gatech/cs1331/code}

\usepackage{colortbl}

% If you have a file called "university-logo-filename.xxx", where xxx
% is a graphic format that can be processed by latex or pdflatex,
% resp., then you can add a logo as follows:

% \pgfdeclareimage[width=0.6in]{coc-logo}{cc_2012_logo}
% \logo{\pgfuseimage{coc-logo}}

\mode<presentation>
{
  \usetheme{Berlin}
  \useoutertheme{infolines}

  % or ...

 \setbeamercovered{transparent}
  % or whatever (possibly just delete it)
}

\usepackage{tikz}
% Optional PGF libraries
\usepackage{pgflibraryarrows}
\usepackage{pgflibrarysnakes}
\usepackage{pgfplots}
\usepackage{fancybox}
\usepackage{listings}
\usepackage[abbr]{harvard}
\usepackage{hyperref}
\hypersetup{colorlinks=true,urlcolor=blue}
\usepackage[english]{babel}
% or whatever

\usepackage[latin1]{inputenc}
% or whatever

\usepackage{times}
\usepackage[T1]{fontenc}
% Or whatever. Note that the encoding and the font should match. If T1
% does not look nice, try deleting the line with the fontenc.


\usepackage{listings}

% "define" Scala
\lstdefinelanguage{scala}{
  morekeywords={abstract,case,catch,class,def,%
    do,else,extends,false,final,finally,%
    for,if,implicit,import,match,mixin,%
    new,null,object,override,package,%
    private,protected,requires,return,sealed,%
    super,this,throw,trait,true,try,%
    type,val,var,while,with,yield},
  otherkeywords={=>,<-,<\%,<:,>:,\#,@},
  sensitive=true,
  morecomment=[l]{//},
  morecomment=[n]{/*}{*/},
  morestring=[b]",
  morestring=[b]',
  morestring=[b]""",
}

\usepackage{color}
\definecolor{dkgreen}{rgb}{0,0.6,0}
\definecolor{gray}{rgb}{0.5,0.5,0.5}
\definecolor{mauve}{rgb}{0.58,0,0.82}

% Default settings for code listings
\lstset{frame=tb,
  language=scala,
  aboveskip=2mm,
  belowskip=2mm,
  showstringspaces=false,
  columns=flexible,
  basicstyle={\scriptsize\ttfamily},
  numbers=none,
  numberstyle=\tiny\color{gray},
  keywordstyle=\color{blue},
  commentstyle=\color{dkgreen},
  stringstyle=\color{mauve},
  frame=single,
  breaklines=true,
  breakatwhitespace=true,
  keepspaces=true
  %tabsize=3
}


% If you wish to uncover everything in a step-wise fashion, uncomment
% the following command:

% \beamerdefaultoverlayspecification{<+->}


\begin{document}

\begin{frame}
  \titlepage
\end{frame}


%------------------------------------------------------------------------
\begin{frame}[fragile]{Recursion}


\begin{itemize}
\item A recursive processes or data structure is defined in terms of itself
\item A properly written recursive function must
\begin{itemize}
\item handle the base case, and
\item convergence to the base case.
\end{itemize}
\item Failure to properly handle the base case or converge to the base case (divergence) may result in infinite recursion.
\end{itemize}


\end{frame}
%------------------------------------------------------------------------

%------------------------------------------------------------------------
\begin{frame}[fragile]{The Factorial Function}


A mathematical definition: For a non-negative integer $n$:
\begin{equation*}
    fac(n) = \begin{cases}
               1                   & \text{if } n \le 1\\
               n * fac(n-1)        & \text{otherwise}
           \end{cases}
\end{equation*}

\begin{itemize}
\item This definition tells us what a factorial is.
\item Defined in cases: a base case and a recursive case
\end{itemize}
 Factorial is defined in terms of itself

\end{frame}
%------------------------------------------------------------------------


%------------------------------------------------------------------------
\begin{frame}[fragile]{A Recursive Factorial Function}

\vspace{-.05in}
\begin{quote}
Mathematics provides a rigorous framework for dealing with notions of {\em what is}, computation provides a rigorous framework for dealing with notions of {\em how to}. -- SICP
\end{quote}
\vspace{-.05in}
To translate the mathematical definition of factorial (what a factorial {\it is} into a computational defiinition ({\it how to} compute a particular factorial), we need to
\begin{itemize}
\item identify the base case(s), and
\item figure out how to get our computation to converge to a base case.
\end{itemize}

For factorial, the solution is straightforward:
\begin{lstlisting}[language=Java]
public static int fac(int n) {
    if (n <= 1) {
        return 1;
    } else {
        return n * fac(n - 1);
    }
}
\end{lstlisting}
\vspace{-.05in}
See \link{\code/algorithms/Fac.java}{Fac.java}

\end{frame}
%------------------------------------------------------------------------

%------------------------------------------------------------------------
\begin{frame}[fragile]{The Substitution Model of Function Evaluation}


\begin{itemize}
\item Functions are evaluated in an eval-apply cycle: function arguments are evaluated (which may in turn require function evaluation), then the function is applied to the arguments.
\item The substitution model of evaluation is a tool for understanding function evaluation in general, and recursive processes in particular.
\end{itemize}
Here's {\tt fac(5)}:
\vspace{-.05in}
\begin{lstlisting}[language=Java]
fac(5)
5 * fac(4)
5 * 4 * fac(3)
5 * 4 * 3 * fac(2)
5 * 4 * 3 * 2 * fac(1)
5 * 4 * 3 * 2 * 1
5 * 4 * 3 * 2
5 * 4 * 6
5 * 24
120
\end{lstlisting}


\end{frame}
%------------------------------------------------------------------------

%------------------------------------------------------------------------
\begin{frame}{Activation Records}

\begin{itemize}

\item Recursive subprograms cannot use statically allocated local
  variables, because each instance of the subprogram needs its own
  copies of local variables

\item Most modern languages allocate local variables for functions on
  the run-time stack.

\item The system provides a stack pointer pointing to the next
  available storage space on the stack.

\item Subprogram instances use a frame pointer that points to their
  activation record, or stack frame, which contains its copies of
  local variables

\end{itemize}

\end{frame}
%------------------------------------------------------------------------

%------------------------------------------------------------------------
\begin{frame}[fragile]{Activation Record Example}

Consider this simplified example code (type annotations elided for brevity):
\scriptsize
\begin{columns}
\begin{column}{4cm}
\begin{lstlisting}[language=C]
void main(args) {
  foo();
}
int foo() {
  int r = 3;
  return fac(r);
}
int fac(n) {
  if (n <=1) {
    return 1
  } else {
    return n * fac(n-1)
  }
}
\end{lstlisting}
\end{column}
\small
\begin{column}{6.5cm}
The stack just before {\tt fac} returns with 6:
\begin{tabular}{r|c|}\hline
main frame & args = ... in main \\ \hline
foo frame & r = 3 in foo \\
 & return value (TBD) \\ \hline
fac(3) frame & parameter n = 3 in fac \\
 & return value (TBD) \\ \hline
fac(2) frame & parameter n = 2 in fac \\
 & return value (TBD) \\ \hline
fac(1) frame & parameter n = 1 in fac \\
 & return value (1 by definition) \\ \hline
\end{tabular}

\end{column}
\end{columns}

\normalsize

\end{frame}
%------------------------------------------------------------------------


%------------------------------------------------------------------------
\begin{frame}[fragile]{Stack Overflow}



\begin{itemize}
\item The run-time stack is finite in size.
\item If you put too many activation records on the stack (for example by calling a recursive function with a ``large'' argument), you will overflow the stack.
\end{itemize}

\begin{lstlisting}[language=Java]
$ java Fac 10000
facLoop(10000)=0
Exception in thread "main" java.lang.StackOverflowError
  at Fac.facIter(Fac.java:35)
  at Fac.facIter(Fac.java:38)
  at Fac.facIter(Fac.java:38)
...
\end{lstlisting}

Three ways to deal with this:

\begin{itemize}
\item limit input size (brittle -- how do you know limit on a particular machine?),
\item increase stack size (brittle -- how do you know how big), or
\item replace recursion with iteration.
\end{itemize}

\end{frame}
%------------------------------------------------------------------------

%------------------------------------------------------------------------
\begin{frame}[fragile]{Looping is Imperative Recursion}


\begin{lstlisting}[language=Java]
    public static int facLoop(int n) {
        int factorialAccumulator = 1;
        for (int x = n; x > 0; x--) {
            factorialAccumulator *= x;
        }
        return factorialAccumulator;
    }
\end{lstlisting}

\begin{itemize}
\item The base case is the termination condition for the loop.
\item The loop variable converges to the termination condition.
\item We ``accumulate'' the answer in the loop.
\end{itemize}

Recursive definitions are often more natural, but imperative/iterative definitions often perform better.

\end{frame}
%------------------------------------------------------------------------


%------------------------------------------------------------------------
\begin{frame}[fragile]{Tail Recursion - Recursive Iteration}

\vspace{-.05in}
\begin{lstlisting}[language=Java]
    private static int facTail(int n) {
        return facHelper(n, 1);
    }
    private static int facIter(int n, int accum) {
        if (n <= 1) {
            return accum;
        } else {
            return facIter(n - 1, n * accum);
        }
    }
\end{lstlisting}
\vspace{-.05in}
Tail call optimization creates an iterative, rather than a recursive process:
\vspace{-.05in}
\begin{lstlisting}[language=Java]
    facTail(5);
    facIter(5, 1);
    facIter(4, 5);
    facIter(3, 20);
    facIter(2, 60);
    facIter(1, 120);
    120
\end{lstlisting}
\vspace{-.05in}
Note: Java does not optimize tail calls, but many other languages (including all functional languages) do.

\end{frame}
%------------------------------------------------------------------------

%------------------------------------------------------------------------
\begin{frame}[fragile]{Closing Thoughts}


\begin{itemize}
\item Remember: A properly written recursive function must
\begin{itemize}
\item handle the base case, and
\item convergence to the base case.
\end{itemize}
\item Today we learned recursive processes.   We'll also learn recursive data structures.
\end{itemize}


\end{frame}
%------------------------------------------------------------------------

% %------------------------------------------------------------------------
% \begin{frame}[fragile]{}


% \begin{lstlisting}[language=Java]

% \end{lstlisting}

% \begin{itemize}
% \item
% \end{itemize}


% \end{frame}
% %------------------------------------------------------------------------


\end{document}
