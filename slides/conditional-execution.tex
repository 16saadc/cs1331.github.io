\documentclass{beamer}

\newcommand{\lesson}{Conditional Execution}


\newcommand{\course}{Introduction to Object-Oriented Programming}
\subject{\course}
\title[\lesson]{\course}
\subtitle{\lesson}

\author[CS 1331]
{Christopher Simpkins \\\texttt{chris.simpkins@gatech.edu}}
\institute[Georgia Tech]

\date[]{}

\newcommand{\link}[2]{\href{#1}{\textcolor{blue}{\underline{#2}}}}
\newcommand{\code}{http://www.cc.gatech.edu/~simpkins/teaching/gatech/cs1331/code}

\usepackage{colortbl}

% If you have a file called "university-logo-filename.xxx", where xxx
% is a graphic format that can be processed by latex or pdflatex,
% resp., then you can add a logo as follows:

% \pgfdeclareimage[width=0.6in]{coc-logo}{cc_2012_logo}
% \logo{\pgfuseimage{coc-logo}}

\mode<presentation>
{
  \usetheme{Berlin}
  \useoutertheme{infolines}

  % or ...

 \setbeamercovered{transparent}
  % or whatever (possibly just delete it)
}

\usepackage{tikz}
% Optional PGF libraries
\usepackage{pgflibraryarrows}
\usepackage{pgflibrarysnakes}
\usepackage{pgfplots}
\usepackage{fancybox}
\usepackage{listings}
\usepackage[abbr]{harvard}
\usepackage{hyperref}
\hypersetup{colorlinks=true,urlcolor=blue}
\usepackage[english]{babel}
% or whatever

\usepackage[latin1]{inputenc}
% or whatever

\usepackage{times}
\usepackage[T1]{fontenc}
% Or whatever. Note that the encoding and the font should match. If T1
% does not look nice, try deleting the line with the fontenc.


\usepackage{listings}

% "define" Scala
\lstdefinelanguage{scala}{
  morekeywords={abstract,case,catch,class,def,%
    do,else,extends,false,final,finally,%
    for,if,implicit,import,match,mixin,%
    new,null,object,override,package,%
    private,protected,requires,return,sealed,%
    super,this,throw,trait,true,try,%
    type,val,var,while,with,yield},
  otherkeywords={=>,<-,<\%,<:,>:,\#,@},
  sensitive=true,
  morecomment=[l]{//},
  morecomment=[n]{/*}{*/},
  morestring=[b]",
  morestring=[b]',
  morestring=[b]""",
}

\usepackage{color}
\definecolor{dkgreen}{rgb}{0,0.6,0}
\definecolor{gray}{rgb}{0.5,0.5,0.5}
\definecolor{mauve}{rgb}{0.58,0,0.82}

% Default settings for code listings
\lstset{frame=tb,
  language=scala,
  aboveskip=2mm,
  belowskip=2mm,
  showstringspaces=false,
  columns=flexible,
  basicstyle={\scriptsize\ttfamily},
  numbers=none,
  numberstyle=\tiny\color{gray},
  keywordstyle=\color{blue},
  commentstyle=\color{dkgreen},
  stringstyle=\color{mauve},
  frame=single,
  breaklines=true,
  breakatwhitespace=true,
  keepspaces=true
  %tabsize=3
}


% \beamerdefaultoverlayspecification{<+->}


\begin{document}

\begin{frame}
  \titlepage
\end{frame}

%------------------------------------------------------------------------
\begin{frame}[fragile]{Structured Programming}


In reasoning about the flow of control to and from a statement, consider control flow issues:
\begin{itemize}
\item Multiple vs. single entry ("How did we get here?")
\item Multiple vs. single exit ("Where do we go from here?")
\item {\tt goto} considered harmful ({\tt goto} makes it hard to answer questions above)
\end{itemize}

Structured programming: block structure, single entry, single exit, no {\tt goto}.  All algorithms expressed by:
\begin{itemize}
\item Sequence - one statement after another
\item Selection - conditional execution (not conditional jumping)
\item Iteration - loops
\end{itemize}

Today we'll learn Java's support for conditional execution

\end{frame}
%------------------------------------------------------------------------


%------------------------------------------------------------------------
\begin{frame}[fragile]{Boolean Values}


There are 10 kinds of people:
\pause
\begin{itemize}
\item Those who know binary,
\pause
\item and those who don't.
\end{itemize}


\end{frame}
%------------------------------------------------------------------------

%------------------------------------------------------------------------
\begin{frame}[fragile]{Boolean Values}


In Java, boolean values have the {\tt boolean} type.  Four kinds of boolean expressions:
\begin{itemize}
\item {\tt boolean} literals: {\tt true} and {\tt false}
\item {\tt boolean} variables
\item expressions formed by combining non-{\tt boolean} expressions with comparison operators
\item expressions formed by combining {\tt boolean} expressions with logical operators
\end{itemize}


\end{frame}
%------------------------------------------------------------------------


%------------------------------------------------------------------------
\begin{frame}[fragile]{Boolean Expressions Formed From Comparisons}


Simple boolean expressions formed with comparison operators:
\begin{itemize}
\item Equal to: {\tt ==}, like $=$ in math
  \begin{itemize}
    \item Remember, {\tt =} is assignment operator, {\tt ==} is comparison operator!
  \end{itemize}
\item Not equal to: {\tt !=}, like $\ne$ in math
\item Greater than: {\tt >}, like $>$ in math
\item Greater than or equal to: {\tt >=}, like $\ge$ in math
\item ...
\end{itemize}
Examples:
\begin{lstlisting}[language=Java]
1 == 1 // true
1 != 1 // false
1 >= 1 // true
1 > 1  // false
\end{lstlisting}

\end{frame}
%------------------------------------------------------------------------

%------------------------------------------------------------------------
\begin{frame}[fragile]{Boolean Expressions Formed From Logical Operators}


Simple boolean expressions can be combined to form larger expressions using:
\begin{itemize}
\item And: {\tt \&\&}, like $\land$ in math
\item Or: {\tt ||}, like $\lor$ in math
\end{itemize}
Examples:
\begin{lstlisting}[language=Java]
(1 == 1) &&  (1 != 1) // false
(1 == 1) ||  (1 != 1) // true
\end{lstlisting}

Also, unary negation operator {\tt !}:

\begin{lstlisting}[language=Java]
!true     // false
!(1 == 2) // true
\end{lstlisting}


\end{frame}
%------------------------------------------------------------------------

%------------------------------------------------------------------------
\begin{frame}[fragile]{The {\tt if-else} Statement}


Conditional execution:
\begin{lstlisting}[language=Java,escapechar=`]
if (`{\it booleanExpression}`)
    // a single statement executed when booleanExpression is true
else
    // a single statement executed when booleanExpression is false
\end{lstlisting}
\vspace{-.1in}
\begin{itemize}
\item {\it booleanExpression} must be enclosed in parentheses
\item {\tt else} not required
\end{itemize}

Example:
\begin{lstlisting}[language=Java]
if ((num % 2) == 0)
    System.out.printf("I like %d.%n", num);
else
    System.out.printf("I'm ambivalent about %d.%n", num);
\end{lstlisting}

\end{frame}
%------------------------------------------------------------------------

%------------------------------------------------------------------------
\begin{frame}[fragile]{Ternary If-Else Expression}


The ordinary {\tt if-else} control structure is a statement, leading
to conditional assignment code like this:
\begin{lstlisting}[language=Java]
String dinner = null;
if (temp > 60) {
    dinner = "grilled";
} else {
    dinner = "baked";
}
\end{lstlisting}

The ternary operator combines the above into one expression (recall
that expressions have values):

\begin{lstlisting}[language=Java]
String dinner = (temp > 60) ? "grilled" : "baked"
\end{lstlisting}

\end{frame}
%------------------------------------------------------------------------


%------------------------------------------------------------------------
\begin{frame}[fragile]{Blocks}


Java is block-structured.  You can enclose any number
of statements in curly braces (\{ ... \}) to create a block.  Blocks
are like single statements (not expressions - they don't have values).
\begin{lstlisting}[language=Java]
if ((num % 2) == 0) {
    System.out.printf("%d is even.%n", num);
    System.out.println("I like even numbers.");
} else {
    System.out.printf("%d is odd.%n", num);
    System.out.println("I'm ambivalent about odd numbers.");
}
\end{lstlisting}
The Java conventions recommend using braces always, even for single
statements.  A very common error is adding statements to an if-branch
and forgetting to add braces.

\end{frame}
%------------------------------------------------------------------------

%------------------------------------------------------------------------
\begin{frame}[fragile]{Multi-way {\tt if-else} Statements}


This is hard to follow:
\begin{lstlisting}[language=Java]
if (color.toUpperCase().equals("RED")) {
    System.out.println("Redrum!");
} else {
    if (color.toLowerCase().equals("yellow")) {
        System.out.println("Submarine");
    } else {
        System.out.println("A Lack of Color");
    }
}
\end{lstlisting}
\vspace{-.05in}
This multi-way {\tt if-else} is equivalent, and clearer:
\vspace{-.05in}
\begin{lstlisting}[language=Java]
if (color.toUpperCase().equals("RED")) {
    System.out.println("Redrum!");
} else if (color.toLowerCase().equals("yellow")) {
    System.out.println("Submarine");
} else {
    System.out.println("A Lack of Color");
}
\end{lstlisting}

\end{frame}
%------------------------------------------------------------------------

%------------------------------------------------------------------------
\begin{frame}[fragile]{Short-Circuit Evaluation}

Here's a common idiom for testing an operand before using it:
\begin{lstlisting}[language=Java]
if ((kids !=0) && ((pieces / kids) >= 2))
    System.out.println("Each kid may have two pieces.");
\end{lstlisting}

In this example Java uses short-circuit evaluation.  If
\begin{quote}
{\tt kids !=0}\\
\end{quote}
evaluates to {\tt false}, then the second sub-expression is not evaluated, thus avoiding a divide-by-zero error.

\vspace{.1in}
Note: You can force a complete evaluation by using {\tt \&} or {\tt |}, for example if you have side effects you want to ensure happen in the second expression.  We mention this fact for completeness but implore you not to write such code.\\

See \link{\code/basics/Conditionals.java}{Conditionals.java} for examples.

\end{frame}
%------------------------------------------------------------------------


%------------------------------------------------------------------------
\begin{frame}[fragile]{The {\tt switch} Statement}

Java provides {\tt switch} statement for multi-way branching.

\begin{lstlisting}[language=Java]
switch (expr) {
case 1:
    // executed only when case 1 holds
    break;
case 2:
    // executed only when case 2 holds
case 3:
    // executed whenever case 2 or 3 hold
    break;
default:
    // executed only when other cases don't hold
}
\end{lstlisting}

\begin{itemize}
\item Execution jumps to the first matching case and continues until a {\tt break}, {\tt default}, or {\tt switch} statement's closing curly brace is reached
\item Type of {\tt expr} can be {\tt char}, {\tt int}, {\tt short}, {\tt byte}, or {\tt String}
\item In example above, what is type of {\tt expr}?
\end{itemize}

\end{frame}
%------------------------------------------------------------------------

%------------------------------------------------------------------------
\begin{frame}[fragile]{Avoid the {\tt switch} Statement}

The {\tt switch} statement is error-prone.
\begin{itemize}
\item {\tt switch} considered harmful.  97\% of fall-throughs
  unwanted\footnote{Peter van der Linden, {\it Deep C Secrets}}
\item Anachronism from ``structured assembly language'', a.k.a. C (a {\tt switch} is just a jump table)
\end{itemize}

You can do without the switch statement.  See

\begin{itemize}
\item \link{\code/basics/CharCountSwitch.java}{CharCountSwitch.java} for a {\tt switch} example,
\item \link{\code/basics/CharCountIf.java}{CharCountIf.java} for the same program using an {\tt if} statement in place of the {\tt switch} statement, and
\item  \link{\code/basics/CharCount.java}{CharCount.java} for the same program using standard library utility methods.
\end{itemize}

\end{frame}
%------------------------------------------------------------------------



%------------------------------------------------------------------------
\begin{frame}[fragile]{Closing Thoughts}


\begin{itemize}
\item Conditional execution straightforward, but watch out for side-effects in boolean assignments.
\item Parenthesize your expressions to make them clear to the reader and to Java.
\item Next we'll learn loops, and we'll have all the tools we need to implement any algorithm.\footnote{Actually we already have all the tools we need to implement any algorithm in a functional style, but we need loops for imperative algorithms.}
\end{itemize}


\end{frame}
%------------------------------------------------------------------------


\end{document}
